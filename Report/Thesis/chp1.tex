%-------------------------------------------------------------------------------
% PACKAGES & DOCUMENT CONFIGURATION
%-------------------------------------------------------------------------------

\documentclass[11pt, a4paper, oneside]{article} % A4, 11pt font, one-sided
\usepackage[english]{babel}
\usepackage[utf8]{inputenc} %setlength{parskip}{0.7em}
\usepackage{bibentry}
\usepackage[margin= 1.3in]{geometry}
\usepackage[colorinlistoftodos]{todonotes}
\usepackage{epigraph}
\usepackage{graphicx}
\usepackage{url}
\usepackage[nottoc]{tocbibind}

\begin{document}

%-------------------------------------------------------------------------------
% TITLE
%-------------------------------------------------------------------------------
\begin{center}\huge Weeks 1 --- 2 of \textit{Making Mario Work Hard}
\par \Large Aaron Ceross (ac14580)
\end{center}

%-------------------------------------------------------------------------------
% BODY
%-------------------------------------------------------------------------------

\section{Overview}
This preliminary week has focused on the development of the basic engine.

\section{Coding Style and Tools}

The code is written in C++ 11 and uses the Google style conventions. I am still
learning these myself but slowly coming to grips with it. The

\par

Development is taking place on my laptop (HP SPECS) running Ubuntu 15.04. I am
using the Atom Editor with.

Code must pass cpplint. In addition, code is dynamically analysed with
LLVM/clang. This provides excellent error messages during development and Atom
also has an autocomplete module for C++ using clang. This has been extremely
useful.

\section{Makefile}

Makefiles are a pain at the best of times. I did some stuff.

The result is that there is an extensible, generic means by which

\section{Graphics and Audio}

I have opted to use SFML library instead of SDL. The reason is selfish at the
moment as I simply want to try it. I have used SDL in the past and found it
quite usable. I do want to broaden my horizons with regards graphics however.
\tabularnewline
SFML is quite similar to SDL, although it seems quite focused on C++. I may
switch the modules to SDL in the future if there are significant performance
issues. For the time being, prototyping with SFML is alright.

\section{Challenges}

\subsection{Learning C++}

There are a lot of differences with Java (the only other OOP language I know).
Having just completed a platformer in Java, implementation in C++ seems much
more involved and nuanced. I have

\end{document}
